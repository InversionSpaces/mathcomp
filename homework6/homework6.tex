\documentclass[a4paper,12pt]{article}
\usepackage[utf8x]{inputenc}
\usepackage[english,russian]{babel}
\usepackage{amsmath}
\usepackage{amsbsy}
\begin{document}
\begin{flushleft}
\section{Первый номер}
	Начну нумерацию с нуля. $N_0 = N_1 = N_2 = 1$.

	В каждый следующий месяц количество пар есть количество пар в текущий месяц, сложенное с количеством пар, уже бывших рождёнными три месяца назад, то есть готовых родить в этот месяц.

	Получаем $N_n = N_{n-1} + N_{n-3}$ или $N_n - N_{n-1} - 0 \cdot N_{n - 2} - N_{n-3} = 0$.

	Характеристический многочлен $P(x) = x^3 - x^2 - 1$.
	
	У него страшные корни, из которых два комплексные.

	Если корни равны $x_0, x_1, x_2$, то формула общего члена последовательности примет вид $N_n = C_0 x_0 ^ n + C_1 x_1 ^ n + C_2 x_2 ^ n$, где $C_0, C_1, C_2$ определяются из начальный условий:

	$$C_0 + C_1 + C_2 = N_0$$

	$$C_0 x_0 + C_1 x_1 + C_2 x_2 = N_1$$

	$$C_0 x_0^2 + C_1 x_1^2 + C_2 x_2^2 = N_2$$
\section{Второй номер}
	Каждый вызов функции бинарного поиска выполняет константное количество операций, т.е. имеет сложность $O(1)$, и вызывает одного потомка с вдвое меньшим промежутком, т.е. с вдвое меньшим количеством входных данных. Вызов функции от $1$ имеет констаное количество операций и не вызывает потомков. То есть:

	$$ T(n) = \left\{ 
	\begin{array}{ll}
		T \left( \dfrac{n}{2} \right) + O(1) & n > 1 \\
		O(1) & n = 1 \\
	\end{array}
	\right\} $$

	То есть $a=1, b=2, c=0=\log_b{a}$. Тогда по Мастер Теореме $T(n)=O(\log{n})$
\end{flushleft}
\begin{center}
Made with \LaTeX{}, Vim and pain
\end{center}
\end{document}
